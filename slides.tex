\documentclass{ctexbeamer}
\usetheme{focus}
\usepackage{amsmath}

\title{递推方程}
\author{李琦煜 \\ 杨思祺 \\ 张远鹏}
\institute{算法设计与分析2020小班17}
\date{2020年2月28日}

\begin{document}
    \begin{frame}
        \maketitle
    \end{frame}

    \begin{frame}{递推方程的求解方法}
        \begin{itemize}
            \setlength{\itemsep}{1em}
            \item 迭代法
            \begin{itemize}
                \setlength{\itemsep}{0.3em}
                \item 直接迭代
                \item 换元迭代
                \item 差消迭代
                \item 递归树
            \end{itemize}
            \item 主定理(Master Theorem)
            \item 尝试法
            \item 递推过程的归纳证明
        \end{itemize}
    \end{frame}

    
    \begin{frame}{迭代法}
            \begin{itemize}
                \item<1-> 直接迭代:插入排序最坏情况下时间分析
                $$W(n)=W(n-1)+n-1,W(1)=0\Rightarrow W(n)=\frac{n(n-1)}{2}$$
                \item<2-> 换元迭代:二分归并排序最坏情况下时间分析
                $$W(n)=2W(n/2)+n-1,W(1)=0$$
                设 $n=2^k$,则有
                \begin{align*} W(n) &=
2^kW(1)+k2^k-(1+2+...+2^{k-1})\\&=k2^k-2^k+1=n\log n-n+1\end{align*}
                \item<3->  使用迭代法得到的解(可能)需要通过数学归纳法验证
            \end{itemize}
    \end{frame}

    \begin{frame}{迭代法}
        \begin{itemize}
            \item 差消迭代:快速排序平均时间分析\\
            $T(n)=\frac{2}{n}\sum\limits_{i=1}^{n-1}T(i)+cn, n\geq 2, T(1) = 0$
            \begin{align*}
            nT(n)&=2\sum_{i=1}^{n-1}T(i)+cn^2\\
            (n-1)T(n-1)&=2\sum_{i=1}^{n-2}T(i)+c(n-1)^2\\
            \text{两式相减得} nT(n)&=(n+1)T(n-1)+2cn-c\\
            \text{故} \frac{T(n)}{n+1}&=\frac{T(n-1)}{n}+\frac{2cn-c}{n(n+1)}\\
            &=2c[\frac{1}{n+1}+...+\frac{1}{3}]+\frac{T(1)}{2}-O(\frac{1}{n})=\Theta(\log n)
            \end{align*}
            故$T(n)=\Theta(n\log n)$
        \end{itemize}
    \end{frame}

    \begin{frame}{练习:求解递推方程}
    \begin{itemize}
        \setlength{\itemsep}{2em}
    \item 
    求解递推方程$T(n)=2T(\sqrt{n})+\log n$
    \pause
    \item 令$m=\log n$, $T(2^m)=2T(2^{m/2})+m$
    \item 令$S(m)=T(2^m)$, 则有$S(m)=2S(m/2)+m=\Theta(m\log m)$
    \item $T(n)=T(2^m)=S(m)=\Theta(m\log m) = \Theta(\log n \log\log n)$
    \end{itemize}
    \end{frame}

    \begin{frame}{迭代模型:递归树}
    \begin{itemize}
    \item 可以处理子问题规模不同的情况
    $$T(n)=T(n/3)+T(2n/3)+n$$
    \begin{center}\includegraphics[width=0.55\textwidth]{2.png}\end{center}
    层数$k \leq \log_{3/2}n$, 故$T(n)=O(n\log n)$
    \item 递归树不平衡时可选取最长(短)路径求和求得一个上(下)界
    \end{itemize}
    \end{frame}
    
    \begin{frame}{主定理(Master Theorem)}
    \begin{itemize}
        \setlength{\itemsep}{0.8em}
        \item 处理一类子问题规模相同的递推方程
        \item $T(n)=aT(n/b)+f(n)  (a\geq1,b>1, T(n)\geq0)$
        \begin{itemize}
        \setlength{\itemsep}{0.5em}
        \item $f(n)=O(n^{\log_ba−\varepsilon}),\varepsilon>0 \Rightarrow T(n)=\Theta(n^{\log_ba})$
        \item $f(n)=\Theta(n^{\log_ba}\log^k n),k\geq 0\Rightarrow T(n)=\Theta(n^{\log_b⁡a}\log^{k+1}n)$\\(扩展的第二类情况)
        \item $f(n)=\Omega(n^{\log_ba+\varepsilon}),\varepsilon>0,\exists c<1,\forall n>n_0,af(n/b)\leq cf(n)\Rightarrow T(n)=\Theta(f(n))$
        \end{itemize}
        \item $f(n)=\frac{n^{\log_ba}}{\log n}$等不能归入以上任何一类
        \item $T(n)=27T(n/3)+n^3$等不满足第三类$c$条件,需要用递归树
    \end{itemize}
    \end{frame}

    \begin{frame}{主定理(Master Theorem)的证明}
    \includegraphics[width=0.9\textwidth]{1.png}
    \end{frame}

    \begin{frame}{主定理(Master Theorem)的证明}
    \begin{itemize}
    \item<+-> $T(n)=c_1n^{\log_ba}+\sum\limits_{i=0}^{\log_bn-1}a^if(\frac{n}{b^i})$
    \item<+-> Case 1: $f(n)=O(n^{\log_ba-\varepsilon})$\\
    $\begin{aligned}T(n)&=c_1n^{\log_ba}+O(n^{\log_ba-\varepsilon}\sum\limits_{i=0}^{\log_bn-1}(b^\varepsilon)^i)\\&=c_1n^{\log_ba}+O(n^{\log_ba-\varepsilon}\frac{b^{\varepsilon\log_bn}-1}{b^\varepsilon-1})=\Theta(n^{\log_ba})\end{aligned}$
    \item<+-> Case 2: $f(n)=\Theta(n^{\log_ba}\log^k n)$\\
    $T(n)=c_1n^{\log_ba}\!+\!\Theta(n^{\log_ba}\!\sum\limits_{i=0}^{\log_bn-1}\log^k(\frac{n}{b^i}))\!=\Theta(n\log_ba\!\log^{k+1}n)$
    \item<+-> Case 3: $f(n)=\Omega(n^{\log_ba+\varepsilon}),\exists 0<c<1,\text{使得}af(\frac{n}{b})\leq cf(n)$\\
    $\begin{aligned}T(n)&\leq c_1n^{\log_ba} +\sum\limits_{i=0}^{\log_bn-1}c^if(n)=c_1n^{\log_ba}+f(c)\frac{c^{\log_bn}-1}{c-1}\\&=c_1n^{\log_ba}+\Theta(f(n))=\Theta(f(n))\end{aligned}$
    \end{itemize}
    \end{frame}

    \begin{frame}{主定理(Master Theorem)的证明}
    \begin{itemize}
    \item 下面引入$\lceil\rceil$和$\lfloor\rfloor$将$n$推广到非$b$的幂的情形,即$T(n)=aT(\lceil n/b\rceil)+f(n)$和$T(n)=aT(\lfloor n/b\rfloor)+f(n)$
    \item 设$n_j=\begin{cases}n,\quad if\ j = 0\\ \lceil n_{j-1}/b\rceil,else\end{cases}$, 由于$\lceil x\rceil\leq x+1$, $n_j\leq \frac{n}{b^j}+\sum\limits_{i=0}^{j-1}\frac{1}{b^i}<\frac{n}{b^j}+\sum\limits_{i=0}^{\infty}\frac{1}{b^i}=\frac{n}{b^j}+\frac{b}{b-1}$\\
    $n_{\lfloor\log_bn\rfloor} < \frac{n}{b^{\lfloor\log_bn\rfloor}}+\frac{b}{b-1}<\frac{n}{b^{n-1}}+\frac{b}{b-1}=b+\frac{b}{b-1}=O(1)$\\
    故$T(n)=\Theta(n^{\log_ba})+\sum\limits_{j=0}^{\lfloor\log_bn\rfloor}a^jf(n_j)$\\
    \item Case 3与上述证明类似。对于Case 2, $j\leq \lfloor\log_bn\rfloor \Rightarrow b^j/n\leq 1$\\
    故$f(n_j)< c(\frac{n}{b^j}+\frac{b}{b-1})^{\log_ba})=c(\frac{n}{b^j}(1+\frac{b^j}{n}\cdot\frac{b}{b-1}))^{\log_ba}=c(\frac{n^{\log_ba}}{a^j})(1+\frac{b^j}{n}\cdot\frac{b}{b-1})^{\log_ba}\leq c(\frac{n^{\log_ba}}{a^j})(1+\frac{b}{b-1})^{\log_ba}=O(\frac{n^{\log_ba}}{a^j})$\\
    类似的对于Case 1可证$f(n_j)=O(n^{\log_ba-\varepsilon􏰀􏰇})$$\hfill\blacksquare$ 
    \end{itemize}
    \end{frame}

    \begin{frame}{通用定理(Akra-Bazzi Theorem)}
    对形如$T(n)=\sum\limits_{i=0}^ka_iT(\frac{n}{b_i})+f(n)(a_i\geq 1, b_i>1)$的递推公式,若$\exists!p>0$使得$\sum\limits_{i=0}^k\frac{a_i}{b_i^p}=1$则有
    \begin{itemize} 
        \setlength{\itemsep}{1em}
    \item $f(n)=O(n^{p-\varepsilon}) \Rightarrow T(n) = \Theta(n^p)$
    \item $f(n)=\Theta(n^p\log^kn) \Rightarrow T(n)=\Theta(n^p\log^{k+1}n)$
    \item $f(n)=\Omega(n^{p+\varepsilon})\Rightarrow T(n)=\Theta(f(n))$
    \end{itemize}
    推论1:\quad $\sum\limits_{i=0}^k\frac{a_i}{b_i}=1,f(n)=\Theta(n\log^k n) \Rightarrow T(n)=\Theta(n\log^{k+1} n)$\\
    推论2:\quad $\sum\limits_{i=0}^k\frac{a_i}{b_i}<1,f(n)=\Omega(n) \Rightarrow T(n)=\Theta(f(n))$
    \end{frame}

    \begin{frame}{更加一般的形式\footnote{Proof:\quad\url{https://people.mpi-inf.mpg.de/~mehlhorn/DatAlg2008/NewMasterTheorem.pdf}}
    \footnote{Example:\quad\url{https://www.blogcyberini.com/2017/07/metodo-de-akra-bazzi.html}}}
    $$T(n)=g(n)+\sum\limits_{i=1}^ka_iT(b_in+h_i(n))$$$$\text{其中}a_i>0, 0<b_i<1,|g(n)|\in O(n^c),|h_i(n)|\in O(\frac{n}{\log^2 n})$$
    通解形式为$$T(n)=\Theta(n^p(1+\int_1^n\frac{g(u)}{u^{p+1}du}),\text{其中}p\text{满足}\sum\limits_{i=1}^ka_ib_i^p=1$$
    \end{frame}

    \begin{frame}{例题}
        \begin{itemize}
        \setlength{\itemsep}{1em}
        \item $T(n)=T(n/3)+T(2n/3)+n$\\
        $\frac{a_1}{b_1}+\frac{a_2}{b_2}=1,f(n)=n=\Theta(n)$,由推论1得$T(n)=\Theta(n\log n)$
        \item $T(n)=\begin{cases} n^2+\frac{7}{4}T(\lfloor\frac{1}{2}n\rfloor)+T(\lceil\frac{3}{4}n\rceil), if\ n\geq 4, \\1, else \end{cases}$\\
        \begin{align*}T(n) &\in \Theta(x^p(1+\int_1^x\frac{g(u)}{u^{p+1}}du)) = \Theta(x^2(1+\int_1^x\frac{u^2}{u^3}dx))\\&=\Theta(x^2(1+\ln x))=\Theta(x^2\log x)\end{align*}
        \end{itemize}
    \end{frame}

    \begin{frame}{尝试法: 猜测解的形式}
    \begin{itemize}
        \item “没有办法的办法”
        \item $T(n)=\frac{2}{n}\sum\limits_{i=1}^{n-1}T(i)+\Theta(n)$
        \begin{enumerate}
        \item $T(n)=C, RHS=\frac{2}{n}C(n-1)+O(n)=2C-\frac{2C}{n}+\Theta(n)$
        \item $T(n)=cn, RHS=\frac{2}{n}\sum\limits_{i=1}^{n-1}ci+\Theta(n)=cn-c+\Theta(n)$
        \item $T(n)=cn^2, RHS=\frac{2}{n}\sum\limits_{i=1}^{n-1}ci^2+\Theta(n)=\frac{2}{n}[\frac{cn^3}{3}+\Theta(n^2)]+\Theta(n)=\frac{2}{3}n^2+\Theta(n)$
        \item $T(n)=cn\log n, RHS=\frac{2c}{n}\sum\limits_{i=1}^{n-1}i\log i+\Theta(n)=cn\log n+\Theta(n)$\\
        注:$\sum\limits_{i=1}^{n-1}i\log i$可用积分逼近(上下界)
        \end{enumerate}
        \item 尝试(猜测)递推方程的解应用归纳法严格证明
    \end{itemize}
    \end{frame}

    \begin{frame}{递推方程的严格归纳证明}
    \begin{itemize} 
        \setlength{\itemsep}{1em}
        \item 归纳法求解递推方程的一般步骤
        \begin{itemize}
        \setlength{\itemsep}{0.8em}
        \item 猜测解的形式
        \item 用数学归纳法证明
            \begin{itemize}\item 找出使解有效的常数\end{itemize}
        \item 确定常数使边界条件成立
        \end{itemize}
        \item 常用技巧
        \begin{itemize}\item 通过引入低阶项获得更紧的解的形式\end{itemize}
    \end{itemize}
    \end{frame}

    \begin{frame}{递推方程的归纳证明}
    \begin{itemize}
        \setlength{\itemsep}{1em}
        \item $T(n)=4T(n/2)+n$,证明$T(n)=O(n^2)$
        \item 假设对于所有的$k<n$,$T(k)\leq ck^2$\\
        $$T(n)=4T(n/2)+n\leq cn^2+n=O(n^2)$$
        \item 伪证,必须证明完全相同的形式
    \end{itemize}
    \end{frame}

    \begin{frame}{更紧的上界}
    \begin{itemize}
        \setlength{\itemsep}{1em}
        \item 加强归纳假设:减去一个低阶项
        \item 假设对于所有的$k<n$,$T(k)\leq c_1k^2-c_2k$\\
        \begin{align*}T(n)&=4T(n/2)+n\\&\leq c_1n^2-2c_2n+n\\&=c_1n^2-c_2n-(c_2n-n) \\ &\leq c_1n^2-c_2n \text{当}c_2>1\text{时}\end{align*}
        \item 取$c_1$足够大来处理初始情况
        \item 再证$T(n)=\Omega(n^2)$,可得$T(n)=\Theta(n^2)$
    \end{itemize}
    \end{frame}

    % \begin{frame}{常系数线性递推方程}
    % \begin{itemize}
    %     \setlength{\itemsep}{1em}
    %     \item $T(n)=c_1T(n−1)+c_2T(n−2)+...+c_kT(n-k)$
    %     \item 特征方程为$x^k-c_1x^{k-1}-...-c_k=0$
    %     \item 确定通解的形式,并根据系数关系确定常数
    % \end{itemize}
    % \end{frame}
\end{document}
