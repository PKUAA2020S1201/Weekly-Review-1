\documentclass{ctexbeamer}
\usetheme{focus}
\usepackage{amsmath}
\usepackage{algorithm}
\usepackage{algorithmicx}
\usepackage{algpseudocode}

\title{分治策略}
\author{李琦煜 \\ 杨思祺 \\ 张远鹏}
\institute{算法设计与分析2020小班17}
\date{2020年2月28日}

\begin{document}

    \begin{frame}
        \maketitle
    \end{frame}

    \begin{frame}
        \frametitle{基本思想}

        设$P$是待求解的问题,$|P|$代表该问题的规模,一般的分治算法思路:

        \begin{itemize}
            \item 如果$|P|$不超过$c$,则直接求解。
            \item 如果$|P|$超过$c$,则将$P$划分为子问题$P_1,P_2,\ldots,P_k$,递归地依次求解并归并得到$P$的答案。
        \end{itemize}

        通常都是递归算法,时间复杂度分析往往依赖于求解递推方程。

    \end{frame}

    \begin{frame}
        \frametitle{分析方法}

        \begin{itemize}
            \item $T(n) = \sum_{i=1}^{k}{a_i T(n-i)} + f(n)$
            \item $T(n) = a T(\frac{n}{b}) + d(n)$
        \end{itemize}

        第一类如汉诺塔分治算法,可以使用迭代、递归树、尝试法等求解。

        第二类如二分检索和归并排序算法,可以使用迭代法、递归树、主定理等求解。

    \end{frame}

    \begin{frame}
        \frametitle{芯片测试}

        有$n$个芯片,其中好芯片比坏芯片至少多$1$片,需要通过测试从中找出$1$片好芯片。测试需要$2$片芯片互相测试,好芯片的报告是正确的,坏芯片的报告是不可靠的。请使用最少的测试次数找出$1$片好芯片。

    \end{frame}

    \begin{frame}
        \frametitle{芯片测试}

        \begin{itemize}
            \item 如果剩下芯片数$k$为偶数,则分为$\frac{k}{2}$两两测试并按照如下规则筛选芯片
                \begin{itemize}
                    \item 如果两片芯片报告都为好,则任取一片。
                    \item 否则两片芯片全都丢弃。
                \end{itemize}
            \item 如果剩下芯片数$k$为奇数,则分为$\lfloor \frac{k}{2} \rfloor$和单独的一片,组内两两测试并按上述规则筛选,单独的一片和其他所有芯片测试
                \begin{itemize}
                    \item 如果报告为坏的次数多于报告为好的次数,则丢弃。
                    \item 如果报告为好的次数不少于报告为坏的次数,则该芯片为好芯片。
                \end{itemize}
            \item 如果剩下的芯片数不超过$3$,可以直接出解。
        \end{itemize}

        每轮至少筛去一半芯片,总时间复杂度$T(n) = T(\frac{n}{2}) + O(n) = O(n)$。

    \end{frame}

    % TODO 快速排序

    \begin{frame}
        \frametitle{幂乘算法}

        计算$a^n$,其中$n$为自然数。

        朴素做法需要$O(n)$次乘法。采用如下分治算法只需要$O(\log n)$次

        \begin{itemize}
            \item $a^n = a^{\frac{n}{2}} \times a^{\frac{n}{2}}$
            \item $a^n = a^{\frac{n-1}{2}} \times a^{\frac{n-1}{2}} \times a$
        \end{itemize}

        该分治算法不仅可以运用在求实数的幂的问题,还可以运用在求矩阵的幂、求整数乘积、求K步最短路等问题。
        
    \end{frame}

    \begin{frame}
        \frametitle{幂乘算法(应用)}

        求Fibonacci数列的第$n$项$F_n$。

        朴素算法需要做$O(n)$次加法,但是得益于

        $$
        \begin{bmatrix}
            F_{n+1} & F_{n} \\
            F_{n} & F_{n-1} \\
        \end{bmatrix}
        =
        \begin{bmatrix}
            1 & 1 \\
            1 & 0 \\
        \end{bmatrix} ^ n
        $$

        可以通过计算rhs求解$F_n$,而该过程可以应用幂乘算法,只需要做$O(\log n)$次乘法。

    \end{frame}

    % TODO 改进分治算法:减少子问题个数

    % TODO 改进分治算法:增加预处理

    \begin{frame}
        END
    \end{frame}
    
\end{document}
